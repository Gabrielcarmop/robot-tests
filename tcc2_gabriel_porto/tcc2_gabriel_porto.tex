% Exemplo de dissertação do INF-UFG com texto em portugues formatado com LaTeX
\documentclass[relatorio,brazil,portuguese,abnt]{inf-ufg}
% Opções da classe inf-ufg (ao usar mais de uma, separe por vírgulas)
%   [tese]         -> Tese de doutorado.
%   [dissertacao]  -> Dissertação de mestrado (padrão).
%   [monografia]   -> Monografia de especialização.
%   [relatorio]    -> Relatório final de graduação.
%   [abnt]         -> Usa o estilo "abnt-alf" de citação bibliográfica.
%   [nocolorlinks] -> Os links de navegação no texto ficam na cor preta.
%                     Use esta opção para gerar o arquivo para impressão
%                     da versão final do seu texto!!!

%----------------------------------------------------- INICIO DO DOCUMENTO %
%----------------------------------------------------- INICIO DO DOCUMENTO %
\usepackage[alf]{abntex2cite}
\usepackage{multirow}

\newcommand{\citep}[1]{\protect\citeauthoronline{#1}~(\citeyear{#1})\xspace}
\newcommand{\rev}[1]{\protect{\color{green}#1\xspace}}

\usepackage{float}
\usepackage{graphicx,color}
\usepackage[brazil]{babel}
\usepackage[T1]{fontenc}
\usepackage{ae}
\usepackage{pgf}
\usepackage{listings}
\usepackage{xspace}
\usepackage[ruled]{algorithm2e}
\usepackage{longtable}
\usepackage{lscape}


%Usado para incluir comentarios
\newcommand{\etal}{{\emph{et al.}\xspace}}

\newcommand{\sigla}{{TFSE}\xspace}

% Controlar linhas orfas e viuvas
\clubpenalty=10000
\widowpenalty=10000
\displaywidowpenalty=10000


\begin{document}
%\selectlanguage{portuguese}

%------------------------------------------ AUTOR, TÍTULO E DATA DE DEFESA %
\autor{Gabriel Porto do Carmo} % (José da Silva)
\autorR{Carmo, Gabriel Porto do} % (da Silva, José)

\titulo{Geração Automática de Casos de Teste para Robot Framework: 
Uma Abordagem Baseada em Inteligência Artificial}
%\subtitulo{\textless Subtítulo do Trabalho\textgreater}

\cidade{Goiânia} % Nome da cidade em foi desenvolvido o trabalho
\dia{10} %
\mes{Dezembro} % Data da apresentação/defesa do trabalho
\ano{2025} % Formato numérico: \dia{01}, \mes{01} e \ano{2009}

%-------------------------------------------------------------- ORIENTADOR %
\orientador{Gilmar Ferreira Arantes}
\orientadorR{Arantes, Gilmar Ferreira}
% Use os comandos a seguir se for Orientadora e nao Orientador.
%\orientadora{\textless Nome da Orientadora\textgreater}
%\orientadoraR{\textless Nome Reverso da Orientadora\textgreater}

%-------------------------------------------------- INSTITUIÇÃO E PROGRAMA %
\universidade{Universidade Federal de Goiás} % {Universidade Federal de Goiás}
\uni{UFG}         % UFG
\unidade{Instituto de Informática} %Instituto de Informática

\universidadeco{\textless Nome da Universidade do Co-orientador\textgreater}
\unico{\textless Sigla da Universidade do Co-orientador\textgreater}
\unidadeco{\textless Nome da Unidade Acadêmica do Co-orientador\textgreater}

\programa{Bacharelado em Ciência da Computação} % Computação
\concentracao{Qualidade de Software}

%-------------------------------------------------- ELEMENTOS PRÉ-TEXTUAIS %
\capa    % Gera o modelo da capa externa do trabalho
\publica % Gera a autorização para publicação em formato eletrônico
\rosto   % Primeira folha interna do trabalho

\input{./pre/pre_aprovacao}
\direitos{Graduando em Ciência da Computação pela Universidade Federal de Goiás, iniciou sua trajetória profissional como estagiário na área de desenvolvimento de software, atuando principalmente com Java e tecnologias relacionadas ao ecossistema da linguagem. Participou de atividades de implementação de funcionalidades, correção de falhas e manutenção de aplicações corporativas. Atualmente trabalha na área de tecnologia, voltado ao desenvolvimento e aprimoramento de sistemas, com interesse crescente em automação de testes, integração contínua e qualidade de software.}



\begin{dedicatoria}
	Dedico este trabalho à minha família, pelo amor, apoio e incentivo em todos os momentos da minha vida.
\end{dedicatoria}

\begin{agradecimentos}
	Agradeço primeiramente aos meus pais, por todo amor, apoio e dedicação ao longo da minha vida. Cada conquista minha é reflexo do esforço, dos valores e da força que sempre me transmitiram.
	
	Aos meus irmãos, deixo um agradecimento especial por estarem ao meu lado em todos os momentos, compartilhando companheirismo, incentivo e aprendizado. O apoio sincero de vocês e a força que me ofereceram, muitas vezes quando eu nem sabia que precisava, foram fundamentais em toda a minha trajetória.
	
	Agradeço também à minha namorada, pela compreensão, paciência e apoio constante durante esta etapa. Sua presença e incentivo foram essenciais para que eu pudesse seguir firme até a conclusão deste trabalho.
	
	Ao meu orientador, expresso meu sincero agradecimento pelas orientações, pela confiança e por todo o conhecimento compartilhado. Sua contribuição foi indispensável para a execução e o amadurecimento deste projeto.
	
	Por fim, agradeço ao restante da minha família, que sempre torceu por mim e ofereceu suporte incondicional. A todos que contribuíram de alguma forma para esta caminhada, deixo registrado meu profundo reconhecimento e gratidão.
\end{agradecimentos}





\epigrafe{“Dificuldades preparam pessoas comuns para destinos extraordinários.”}
{C. S. Lewis}
{Mere Christianity}

\chaves{ Automação de Testes; Inteligência Artificial; Robot Framework; Selenium WebDriver; GitHub Actions}

\begin{resumo}
	Este trabalho apresenta o desenvolvimento e a avaliação de uma solução de automação de testes funcionais integrada a mecanismos de inteligência artificial. A proposta utiliza Robot Framework e Selenium WebDriver para realizar a detecção automática de falhas em aplicações web, complementada por um módulo de análise inteligente que interpreta erros durante a execução dos testes por meio da API Gemini. Além disso, a solução integra diretamente com o GitHub Actions e GitHub Issues, permitindo a abertura automática de registros de falhas com evidências detalhadas, como capturas de tela, logs e hipóteses geradas pela IA. A pesquisa tem caráter aplicado e explora como a combinação entre automação, inteligência artificial e práticas de integração contínua pode aprimorar a rastreabilidade, a eficiência e a confiabilidade no processo de garantia da qualidade de software. 
\end{resumo}



\keys{Test Automation; Artificial Intelligence; Robot Framework; Selenium WebDriver; Continuous Integration}

\begin{abstract}{Automatic Test Case Generation for Robot Framework: An Artificial Intelligence–Based Approach}
	This work presents the development and evaluation of a functional test automation solution integrated with artificial intelligence mechanisms. The proposed approach employs Robot Framework and Selenium WebDriver to perform automatic failure detection in web applications, complemented by an intelligent analysis module that interprets errors during test execution using the Gemini API. Additionally, the solution integrates directly with GitHub Actions and GitHub Issues, enabling the automatic creation of defect reports containing detailed evidence such as screenshots, logs, and AI-generated diagnostic hypotheses. This applied research explores how the combination of test automation, artificial intelligence, and continuous integration practices can enhance traceability, efficiency, and reliability within the software quality assurance process.
\end{abstract}



\tabelas[figtab]
%Opções:
%nada [] -> Gera apenas o sumário
%fig     -> Gera o sumário e a lista de figuras
%tab     -> Sumário e lista de tabelas
%alg     -> Sumário e lista de algoritmos
%cod     -> Sumário e lista de códigos de programas
%
% Pode-se usar qualquer combinação dessas opções.
% Por exemplo:
%  figtab       -> Sumário e listas de figuras e tabelas
%  figtabcod    -> Sumário e listas de figuras, tabelas e códigos de programas
%  figtabalg    -> Sumário e listas de figuras, tabelas e algoritmos
%  figtabalgcod -> Sumário e listas de figuras, tabelas, algoritmos e
%                  códigos de programas

%--------------------------------------------------------------- CAPÍTULOS %
%\chapter{Introdu\c{c}\~ao - Tema e Problematiza\c{c}\~ao}
\chapter{Introdução}
\label{cap:Intro}

\par Os testes de software são essenciais para a garantia e validação das funcionalidades de um projeto, desde a concepção inicial até sua versão final. Eles terão como objetivo validar funcionalidades, identificar falhas e assegurar a qualidade do software, evitando que erros cheguem ao usuário.
\par No âmbito dos testes, existirão diversos tipos e técnicas que poderão ser aplicados de acordo com as funcionalidades e a estrutura do projeto. Os testes poderão ser manuais ou automatizados, abrangendo desde testes unitários, que validarão individualmente uma parte do código, até testes de integração e de sistema, que avaliarão a interação entre módulos e o funcionamento completo do software. Cada abordagem terá seu papel na construção de um processo confiável de desenvolvimento e entrega ao usuário.
\par Em alguns casos, os testes tornam-se ainda mais indispensáveis, especialmente quando envolvem funcionalidades críticas que podem comprometer completamente o software e causar falhas catastróficas no mundo real. Nesse cenário, a validação contínua da qualidade e da segurança torna-se uma exigência constante. Para isso, os processos e as técnicas de teste devem ser cuidadosamente planejados, priorizando as melhores estratégias para cobrir todas as funcionalidades essenciais do software.
\par Durante o desenvolvimento e a manutenção do software, mudanças e melhorias serão necessárias, e os testes deverão acompanhar essas alterações. Uma das alternativas para esse monitoramento e aperfeiçoamento será a aplicação de testes automatizados, que surgirão como uma solução eficaz para manter o processo de validação sempre atualizado.
\par Para que essa estratégia seja realmente eficaz, os testes não poderão ser conduzidos como uma etapa isolada ou final do desenvolvimento. Eles precisarão ser planejados desde o início do projeto e integrados ao fluxo de trabalho da equipe, tornando-se parte natural e contínua do ciclo de vida do software. Esse paradigma é fortemente defendido por metodologias ágeis e pela cultura DevOps, que promoverão práticas como Integração Contínua (CI) e Entrega Contínua (CD), estimulando que a validação da qualidade ocorra em todas as fases do desenvolvimento, de forma automatizada e iterativa.
\par Nesse contexto, formula-se a questão que guiará o desenvolvimento deste trabalho:
de que maneira a integração entre automação de testes e mecanismos de inteligência artificial poderá aprimorar o processo de análise de falhas, ampliar a rastreabilidade e tornar o ciclo de garantia da qualidade mais eficiente?
\par Assim, insere-se neste cenário a proposta deste trabalho, que explorará a automação de testes em conjunto com mecanismos modernos de inteligência artificial e pipelines de CI/CD, demonstrando como essas abordagens poderão elevar a confiabilidade, a rastreabilidade e a eficiência do processo de desenvolvimento de software.

\section{Justificativa} 
\label{sec:justif}
\par   Este trabalho surgiu do interesse pessoal do autor, pela área de testes de software e pela busca por soluções que automatizem estes testes. A validação automatizada exige uma abordagem constante e personalizada. O grande desafio é garantir que todas as funcionalidades do software continuem funcionando corretamente, especialmente após alterações no seu código-fonte, o que é fundamental para garantir a qualidade e a segurança do software.
\par Com foco principal na melhoria dos testes de software por meio da automação, este trabalho também utiliza Inteligência Artificial (IA) para validar e apoiar a tomada de decisões. Isso não apenas reduz o risco de problemas não detectados, mas também contribui para um ciclo de desenvolvimento mais ágil e seguro.
\par   Além disso, este trabalho também busca fomentar discussões sobre como a automação de testes e o uso de IA podem melhorar a validação de sistemas de software. Ao detectar defeitos, o quanto antes, agiliza o ciclo de desenvolvimento, beneficiando os envolvidos no processo de desenvolvimento de software.
\par  Em resumo, a ideia é o  aprimoramento contínuo e a ampliação do conhecimento sobre todos os aspectos relacionados à automação de testes, beneficiando assim a comunidade de profissionais e entusiastas da área.

\section{Objetivos}
\label{sec:objtivos}

\subsection{Objetivo Geral}
\label{sub:objGeral}

\par    Esse estudo tem como objetivo analisar como a automação de teste, combinado com o uso de inteligencia artificial podem melhorar o teste de software, com foco na validação contínua das funcionalidades do software, promovendo maior qualidade, segurança e eficiência no ciclo de desenvolvimento.

\subsection{Objetivos Específicos}
\label{sub:objEspec}
\begin{itemize}
    \item   Compreender como a automação de testes pode ser aplicada de forma eficaz em todas as partes de um  software;
    \item   Analisar como a Inteligência Artificial pode ser utilizada para identificar falhas e auxiliar na tomada de decisões durante os ciclos de testes;
    \item   Apresentar casos práticos onde o uso do \textit{Robot Framework}\footnote{https://robotframework.org/} e do \textit{GitHub Actions}\footnote{https://github.com/features/actions} contribui para a detecção de defeitos e a integração contínua do código;
    \item   Avaliar a eficácia da validação automatizada.
\end{itemize}

\section{Metodologia}
\par Este trabalho se caracteriza como uma pesquisa aplicada, de natureza qualitativa e exploratória. Sua abordagem metodológica concentra-se na investigação e experimentação prática de técnicas voltadas à detecção e análise de falhas em sistemas web, utilizando ferramentas de automação de testes e plataformas de versionamento de código como elementos centrais do estudo.
\par Na fase inicial, realizou-se um levantamento teórico sobre conceitos fundamentais de testes de software, automação e qualidade, apoiado em obras clássicas e materiais contemporâneos da área, incluindo Myers \cite{myers2011}, Pressman \cite{pressman2002}, McCabe \cite{mccabe1976}, bem como artigos, guias técnicos e documentações oficiais relacionadas ao Robot Framework, Selenium WebDriver, integração contínua e técnicas funcionais e estruturais de teste. Esse levantamento teve como finalidade fornecer a base conceitual necessária para orientar o desenvolvimento do experimento.
\par  Em seguida, foram desenvolvidos testes automatizados utilizando Robot Framework e \textit{Selenium WebDriver}. Os primeiros testes foram aplicados em um site com cenários de erro proposital\footnote{Disponível em: \href{https://the-internet.herokuapp.com/login}{https://the-internet.herokuapp.com/login}. Acesso em: 22 jun. 2025.}. Esse ambiente foi ideal para validar o funcionamento da estratégia de testes automatizados.
\par  Depois dessa fase inicial, o projeto será implementada em um sistema real para testar sua eficácia em um ambiente de produção. Os testes foram configurados para rodar automaticamente sempre que um novo \textit{commit} é feito no repositório do projeto no GitHub, ou em horários definidos, com o uso do \textit{GitHub Actions}. Quando uma falha é detectada, o sistema gera automaticamente uma \textit{issue} no próprio \textit{GitHub}, contendo \textit{prints} da tela, \textit{logs} e uma descrição resumida com a ajuda de uma inteligência artificial que interpreta o erro.
\par    Por fim,  os resultados são analisados, observando os tipos de erros mais frequentes, o tempo ate encontrar  a falha, e a utilidade das informações geradas para os desenvolvedores.


\section{Organização do Trabalho}
\label{sec:organ}

Para facilitar o alcance dos objetivos especificados, além desta Introdução, o restante deste trabalho está organzado da seguinte forma:

\begin{itemize}
	\item No Capítulo \ref{cap:fundTeo} é apresentada a fundamentação teórica, que embasa todo este trabalho.
	\item No Capítulo \ref{cap:testAut} é apresentado o projeto propriamente dito...
	\item No Capítulo \ref{cap:consFin} são apresentadas as considerações finais e os possíveis desdobramentos futuros deste trabalho.
\end{itemize}



%\chapter{Introdu\c{c}\~ao - Tema e Problematiza\c{c}\~ao}
\chapter{Fundamentação Teórica}
\label{cap:fundTeo}

\par Para esta dissertação, é fundamental apresentar os conceitos essenciais que baseiam o processo de teste de software, destacando sua relevância para assegurar a qualidade em projetos de desenvolvimento. Os testes assumem um papel central na verificação de requisitos e na validação de funcionalidades, permitindo a identificação de falhas e a prevenção de erros que poderiam comprometer tanto a experiência do usuário quanto a segurança do software.
\par  Com base nisso, diferentes técnicas de teste foram desenvolvidas e aprimoradas ao longo do tempo. Entre elas, destacam-se as técnicas funcionais e estruturais, que oferecem visões complementares para a avaliação do software. Enquanto os testes funcionais verificam se o sistema atende às especificações definidas, os estruturais exploram a lógica interna do código, permitindo uma análise mais detalhada do seu comportamento.
\par Já os níveis de teste representam as diferentes etapas em que o software é avaliado ao longo do desenvolvimento. Eles vão desde a verificação de partes menores do sistema até a análise do produto final, pronto para uso. A ideia é que cada nível contribua para aumentar a confiança no software, começando com verificações mais específicas e chegando a testes mais amplos, que simulam o uso real.

% enriquecer mais essa parte 

\section{Técnicas de Teste de Software}
\label{sec:tectest}

\par Com o avanço nas práticas de desenvolvimento de software, diversas novas abordagens de teste foram surgindo. No entanto, algumas técnicas clássicas, desenvolvidas ainda nas primeiras décadas da computação, permanecem fundamentais até os dias de hoje, e com o mesmo propósito, de revelar a presença de defeitos no software.
\par As principais técnicas de teste ganharam destaque a partir das publicações de Glenford Myers \cite{myers2011}, que propôs a divisão entre testes funcionais e estruturais. Essas duas abordagens analisam de forma diferente o software. Os testes funcionais verificam se o software atende aos seus requisitos, enquanto os testes estruturais avaliam o código por dentro. Dessa forma, é possível validar o software de maneira mais completa.
\par Essa relação demonstra que a aplicação de uma única técnica não é suficiente para garantir a segurança e a qualidade do software. É justamente a combinação entre diferentes abordagens que amplia a cobertura dos testes e fortalece a detecção de falhas. Além disso, a consolidação desses conceitos possibilitou o surgimento de métodos derivados e de ferramentas que são amplamente utilizadas em todo o ecossistema de testes.

% enriquecer um pouco mais esta introdução

\subsection{Técnicas de Teste Funcional}
\label{sub:tectestfunc}

\par A técnica de teste funcional ou teste caixa-preta, Figura \ref{fig:testeCxPreta}\footnote{\url{https://www.linkedin.com/pulse/testes-de-caixa-preta-anselmo-n-de-oliveira-ygdgf/}}, se baseia em verificar o funcionamento do software, sem ter que testar necessariamente o código que foi implementado.  Nesse método, são definidos dados de entrada e os resultados esperados para cada situação. O teste é aprovado quando as respostas obtidas correspondem as respostas esperadas.
\par Da forma que a técnica foi desenvolvida, ela consegue ser aplicada em diferentes níveis do software, desde um método isolado até a aplicação completa. Além disso, esse modelo permite ao testador ter uma perspectiva mais próxima da experiência do usuário final, avaliando o comportamento do software como ele seria percebido na prática.

\begin{figure}[!ht]
    \centering
    \includegraphics[width=0.75\linewidth]{./fig/teste-caixa-preta.png}
    \caption{Teste de Caixa Preta}
    \label{fig:testeCxPreta}
\end{figure}

\subsection{Técnicas de Teste Estrutural}
\label{sub:tectestest}

\par A técnica de teste estrutural, Figura \ref{fig:testecxbranca}, também conhecida como teste de caixa-branca, foca na validação do software a partir da análise direta do código-fonte. Diferentemente da abordagem funcional, esse modelo exige que o testador examine as estruturas internas, a lógica de execução e os métodos implementados, garantindo que todos os caminhos possíveis do código sejam devidamente testados.
\par Com base nesse código, são elaborados casos de teste que buscam cobrir toda a execução do requisito avaliado.

\par Os testes de unidade e os testes de integração são exemplos dessa abordagem, sendo amplamente aplicados pelos desenvolvedores ao longo do processo de desenvolvimento. Em um teste de unidade, por exemplo, pode-se verificar se um método responsável por calcular descontos e retorna o valor esperado para diferentes entradas. Já em um teste de integração, a atenção pode estar na interação entre módulos, como a comunicação entre o módulo de autenticação e o módulo de geração de relatórios. Esses testes permitem revelar a presença de defeitos em partes específicas do software e assegurar que cada elemento funcione corretamente em diferentes cenários.

\par É importante lembrar que a técnica estrutural não substitui a técnica funcional, elas devem se complementar.  Enquanto o teste funcional valida o software a partir da visão do usuário,  o estrutural garante confiabilidade e segurança  do código, contribuindo para uma cobertura de testes mais completa e eficaz.

\textcolor{red}{[é necessário informar qua é a fonte da imagem do teste caixa branca. Adicionar ao arquivo bib e referenciar aqui.]}

\begin{figure}[!ht]
    \centering
    \includegraphics[width=0.75\linewidth]{./fig/TesteCaixaBranca.png}
    \caption{Teste de Caixa Branca}
    \label{fig:testecxbranca}
\end{figure}



% Adicionar uma imagem. Enriquecer o teste e apresentar alguns exemplos.

\subsection{Técnicas de Teste Baseada em Defeitos}
\label{sub:tectestdef}

\par A técnica baseada em defeitos é uma abordagem  que visa identificar vulnerabilidades no software por meio da introdução intencional de falhas. Em vez de se basear apenas em requisitos funcionais, essa técnica busca avaliar como o software se comporta diante de erros, explorando suas fragilidades e limites.
Para isso, são utilizadas técnicas como \textit{Mutation Testing}\footnote{\url{https://symflower.com/en/company/blog/2023/using-mutation-testing/}}, conforme Figura \ref{fig:testeMutacao}, que realiza pequenas alterações no código, simulando erros comuns  e \textit{Fuzz Testing}, que fornece entradas aleatórias e inesperadas ao software. O objetivo é verificar se os testes conseguem detectar esses problemas e garantir que o softaware  continue operando corretamente mesmo em situações adversas.

\begin{figure}[!ht]
    \centering
    \includegraphics[width=0.75\linewidth]{./fig/mutant.png}
    \caption{Teste de Mutantes.}
    \label{fig:testeMutacao}
\end{figure}

\subsection{Critérios de Teste}
\label{sub:crittest}

\par Os critérios de testes tem um papel fundamental no processo de validação, sendo responsáveis por determinar as regras e diretrizes que serão utilizadas no processo de analise. Dessa forma, o testador consegue determinar se o conjunto de testes elaborado é suficiente para verificar adequadamente o componente ou funcionalidade em questão, aumentando assim a chance de revelar a presença de defeitos no software.
\par Existem diferentes tipos de critérios, de acordo com a técnica que foi aplicada. Os principais são, o critérios de teste funcional, que focam principalmente no comportamento externo da aplicação, e os critérios de teste estrutural,  que considera a lógica interna do código-fonte. Cada tipo possui técnicas específicas para auxiliar no planejamento e execução dos testes.

\subsubsection{Critérios de Teste Funcional}
\label{subsub:crittestfunc}

\par Para validar se o software atende aos requisitos, os critérios de testes funcional exercem papel fundamental.  Essa abordagem permite que o testador simule diferentes cenários de uso real, avaliando se as funcionalidades respondem corretamente de acordo com o que foi especificado

\begin{enumerate}
    \item \textbf{Particionamento por Equivalência};
\par O particionamento por equivalência é uma técnica que busca simplificar os testes, dividindo os dados de entrada em grupos que devem ser tratados da mesma forma pelo software. A ideia é que, se um valor de um grupo especifico funciona corretamente, os demais também funcionarão, esses grupos são conhecidos como classe de equivalência, e podem conter dados validos e inválidos. 

\begin{figure}[!ht]
    \centering
    \includegraphics[width=0.5\linewidth]{./fig/particao_equivalencia.png}
    \caption{Partição de equivalência}
    \label{fig:partEquiv}
\end{figure}
\textcolor{red}{[é necessário informar qua é a fonte da imagem do critério e teste funcional, particionamento por equivalência. Adicionar ao arquivo bib e referenciar aqui. Ou pode ser também uma nota de rodapé. Isso se aplica a todas as imagens que estão sem referência da fonte onde foram obtidas.]}

    \item \textbf{Análise do Valor Limite};
\par A analise do valor limite, é uma técnica que se baseia em pontos limites de uma faixa de dados, apresentando um abordagem simples, onde os erros costumam ocorrer nos pontos extremos das entradas. A partir disso, em vez de testar os valores típicos, o testador analisa os valores mais altos e baixos da amostra de dados. Essa técnica pode ser considerada um complemento para o particionamento por equivalência.  
%adcionar uma imagem e falar um pouco mais e apresentar exemplos 
% explicar como funciona melhor 
    \item \textbf{Grafo de Causa e Efeito}
\par A analise do valor limite, é uma técnica que se baseia em pontos limites de uma faixa de dados, apresentando um abordagem simples, onde os erros costumam ocorrer nos pontos extremos das entradas. A partir disso, em vez de testar os valores típicos, o testador analisa os valores mais altos e baixos da amostra de dados. Essa técnica pode ser considerada um complemento para o particionamento por equivalência.  \cite{araujo2015}
\begin{figure} [!ht]
    \centering
    \includegraphics[width=0.5\linewidth]{./fig/GrafoDeCausaEfeito.png}
    \caption{Grafo de Causa e Efeito}
    \label{fig:gfc}
\end{figure}
%adcionar uma imagem e falar um pouco mais e apresentar exemplos 
% explicar como funciona melhor 
    \item \textbf{Tabela de Decisão}.
\par A Tabela de decisão é um técnica conhecida por testar softwares que envolvem varias condições lógicas e regras de negocio mais complexas. Ela ajuda a organizar e visualizar de maneira clara como o software deve se comportar com diferentes combinação de entradas.
\par A técnica consiste em apresentar as ações e condições de entrada em um tabela, onde cada linha corresponde a um regra especifica, e as condições são avaliadas como verdadeiras ou falsas. Essa estrutura é bastante útil em cenários com diversas variações de comportamento, pois auxilia na identificação de todos os casos relevantes a serem testados.
\begin{figure}[!ht]
    \centering
    \includegraphics[width=0.5\linewidth]{./fig/TabelaDecisao.png}
    \caption{Tabela de Decisão \cite{araujo2015}}
    \label{fig:tabeladecisao}
\end{figure}

\end{enumerate}
%adcionar uma imagem e falar um pouco mais e apresentar exemplos 
% explicar como funciona melhor 
\subsubsection{Critérios de Teste Estrutural}
\label{subsub:crittestest}

\par Para uma análise mais precisa do código-fonte, são aplicados os critérios de teste estrutural. Eles são utilizados quando o objetivo é verificar o funcionamento interno do software, analisando diretamente a lógica implementada, como condições, fluxos e estruturas de decisão presentes no código.
\par Alguns exemplos comuns desses critérios, referencia são:
        \begin{itemize}
            \item \textbf{Cobertura de instruções} , que garante que cada linha de código seja executada ao menos uma vez. Por exemplo, em um método que soma dois números, o critério assegura que a linha de retorno seja testada em diferentes situações.
        \end{itemize}
\begin{itemize}
    \item \textbf{Cobertura de caminhos} , que busca percorrer todos os caminhos possíveis de execução do código. Esse critério é útil em sistemas com múltiplas regras de negócio encadeadas.
    \item \textbf{Cobertura de decisões} , que verifica se todas as ramificações de um comando condicional, como if ou switch, foram testadas, e se os resultados estão de acordo com o esperado.
\end{itemize}

\par Para viabilizar a aplicação desses critérios, utiliza-se o Grafo de Fluxo de Controle (GFC) constitui uma representação gráfica que ilustra o fluxo de execução de programas ou aplicações, permitindo a análise detalhada do comportamento interno do software. Desenvolvido originalmente por Frances E. Allen, esse grafo é direcionado e orientado a processos, no qual os nós representam blocos básicos de instruções e as arestas indicam os possíveis caminhos de fluxo de controle. Essa estrutura facilita a compreensão, otimização e verificação da lógica de programas, sendo essencial para o desenvolvimento de software mais seguro e eficiente.
%Fazer uma citação
\par Dentre essas métricas,  a Complexidade Ciclomática é uma medida fundamental em Engenharia de Software, introduzida por Thomas McCabe na década de 1970. Seu objetivo principal é quantificar a complexidade de um módulo de software por meio da análise de seus pontos de decisão, os quais determinam a quantidade de caminhos de execução possíveis no código.
%Fazer uma citação
\par Quanto maior o valor da complexidade ciclomática, maior a quantidade de ramificações lógicas (como estruturas condicionais e loops), o que tende a aumentar a dificuldade de compreensão, teste e manutenção do software. Por outro lado, valores mais baixos indicam código mais simples, estruturado e confiável, reduzindo a probabilidade de defeitos e facilitando a evolução do sistema.
%adcionar imagem 

\par Considerando essas características, os testes estruturais são geralmente aplicados nos níveis de teste unitário e de integração. Sua implementação requer conhecimento detalhado do código-fonte e o uso de ferramentas específicas, sendo portanto de responsabilidade primária do desenvolvedor. Essa abordagem assegura que a verificação da qualidade interna do software ocorra desde as etapas iniciais de desenvolvimento, contribuindo significativamente para a construção de sistemas mais robustos e confiáveis.

\par 
%Adicionar complexidade ciclomatica

% Adicionar exemplos de critérios de teste estrutural

\subsection{Níveis de Teste}
\label{sec:nivtest}

Exemplo de Citação de Imagem. Figura \ref{fig:modelo-v}

\begin{figure}[!ht]
    \centering
    \includegraphics[width=0.5\linewidth]{./fig/modelo-v}
    \caption{Modelo V}
    \label{fig:modelo-v}
\end{figure}

\subsubsection{Teste Unitário}
\label{subsub:testunit}

\par O testes unitários representam o nível básicos no processo de software, eles são responsáveis por verificar pequenos trechos de código, como métodos, funções e módulos isolados. A principal função é encontrar erros lógicos ou de implementação  ainda nas etapas iniciais do desenvolvimento

\subsubsection{Teste de Integração}
\label{subsub:testint}

\par Após a validação individual dos módulos, o teste de integração busca encontrar falhas que podem surgir quando esses módulos são integrados. Essa etapa é essencial para garantir que a estrutura do projeto esteja funcional e bem acoplada.

\subsubsection{Teste de Sistema}
\label{subsub:testsist}

\par No teste de sistema, o software como um todo é validado, já em sua versão integrada. Os testes são executados em condições semelhantes às do ambiente real de uso, simulando o comportamento de um usuário final e validando os requisitos de forma geral.

\subsubsection{Teste de Aceitação}
\label{subsub:testaceit}

\par Os testes de aceitação são realizados por um grupo restrito de usuários finais, buscando confirma se o software esta pronto para ser colocado em produção. Esses devem simular tarefas e rotinas que serão realizadas quando o projeto estiver entregue

%Fazer referencia da imagem com o texto

\subsection{Tipos de Teste}
\label{sub:tiposteste}

\subsubsection{Teste de Performance}
\label{subsub:testperform}

 \par O teste de performance tem como objetivo avaliar a eficiência de um software em termos de tempo de resposta, uso de recursos e estabilidade sob diferentes condições de operação. Esse tipo de teste é fundamental para assegurar que o sistema atenda a requisitos de qualidade relacionados a rapidez, confiabilidade e escalabilidade, fatores determinantes para a experiência do usuário e para a competitividade da aplicação no mercado.

\par Por meio do teste de performance, é possível identificar gargalos que impactam diretamente o funcionamento do sistema, como consultas lentas ao banco de dados, consumo excessivo de memória ou problemas de rede. Além disso, esses testes permitem definir parâmetros de aceitação, tais como o tempo máximo aceitável de resposta, a quantidade de transações suportadas por segundo e a eficiência na utilização de recursos do servidor.


\subsubsection{Teste de Carga}
\label{subsub:testecarga}

 \par O teste de carga consiste em simular a demanda real de usuários em um software, com o objetivo de analisar seu comportamento sob diferentes condições de tráfego, desde situações de uso leve até picos de acesso intenso. Esse método é geralmente aplicado nas fases finais do ciclo de desenvolvimento, permitindo avaliar a robustez e a estabilidade do sistema perante cenários operacionais previsíveis. Por meio dessa técnica, é possível assegurar que a aplicação suportará o volume esperado de usuários, além de identificar e corrigir possíveis problemas de desempenho antes que o software seja disponibilizado para o público geral

\subsubsection{Teste de Estresse}
\label{subsub:testestress}

 \par O teste de estresse tem como objetivo avaliar até onde um sistema consegue suportar situações extremas, indo além das condições normais de uso. Ele é essencial porque não basta que um software funcione bem em cenários comuns, é preciso garantir também que ele se mantenha estável quando submetido a sobrecargas ou pressões inesperadas. Esse tipo de teste mostra o quão robusto e confiável o sistema realmente é, revelando se ele resiste sem travar ou falhar quando colocado em seu limite.

\subsubsection{Teste de Regressão}
\label{subsub:testregres}

\par O teste de regressão é uma técnica essencial no desenvolvimento de software, aplicada sempre que uma nova versão do sistema é lançada ou quando ele passa por ciclos de evolução contínua. O objetivo principal dessa abordagem é assegurar que as modificações implementadas não introduzam falhas em funcionalidades que já estavam sólidas.

\par Para isso, todos os testes que foram previamente executados nas versões ou ciclos anteriores são reaplicados à nova versão em avaliação. Além disso, leva-se em consideração as fases e técnicas de teste mais adequadas, de acordo com o impacto causado pelas alterações recentes.


\section{Ferramentas de Teste}
\label{sec:testtools}

\par À medida que sistemas se tornam mais complexos e exigem maior rapidez em sua entrega, o uso de ferramentas de teste passa a ser essencial para apoiar equipes na criação, execução e gestão das atividades de verificação e validação.

\par Essas ferramentas oferecem recursos que vão desde a elaboração de casos de teste até a automação da execução e o acompanhamento dos resultados, possibilitando maior controle sobre o processo e reduzindo falhas que poderiam comprometer a experiência do usuário final. Além disso, permitem padronizar práticas, otimizar o tempo das equipes e fornecer métricas importantes para a tomada de decisão.

\par Outro aspecto importante é a capacidade dessas ferramentas de se integrarem a diferentes fases do ciclo de desenvolvimento, especialmente em metodologias ágeis e em práticas de integração e entrega contínuas (CI/CD). Isso possibilita que a verificação de qualidade seja incorporada de forma contínua e incremental, acompanhando a evolução do software e prevenindo problemas em estágios mais avançados do projeto.

\section{Teste de Software e Inteligência Artificial}
\label{sec:testia}

\par A inteligência artificial está revolucionando a área de testes de software, trazendo um novo patamar de eficiência para processos que antes dependiam quase que exclusivamente do trabalho manual e demandavam tempo significativo.  O uso de técnicas de aprendizado de máquina, mineração de dados e análise preditiva permite acelerar atividades fundamentais como a geração de casos de teste, a priorização de cenários e a detecção de falhas, e ate mesmo propor soluções para os erros encontrados .

\par Entre os benefícios mais relevantes está a capacidade da IA em automatizar a criação de casos de teste, a partir da análise de requisitos, históricos de execução e até do comportamento dos usuários em produção. Isso amplia a cobertura dos testes e reduz a possibilidade de lacunas que poderiam comprometer a confiabilidade do sistema. Além disso, algoritmos inteligentes podem identificar padrões de falhas recorrentes e priorizar a execução dos testes mais críticos, direcionando os recursos para as áreas de maior risco.

\par Outro ponto de destaque é o suporte da IA na gestão de resultados de teste. Em projetos de grande porte, a quantidade de dados gerados pode dificultar a análise manual. Nesse cenário, sistemas baseados em IA auxiliam na categorização automática de falhas, na identificação de falsos positivos e na sugestão de correções mais prováveis, agilizando a tomada de decisão das equipes de desenvolvimento e qualidade.


\chapter{Projeto de Teste Automatizado}
\label{cap:testAut}

\par O projeto em questão, visa demonstrar a aplicação pratica do \textit{Robot Framework}\footnote{\url{https://robotframework.org/}} na automação de testes funcionais em uma aplicação \textit{web} real. A base do trabalho  é avaliar como a utilização de um \textit{framework} de automação baseado em \textit{keywords} pode contribuir para a melhoria do processo de garantia da qualidade de software, ao proporcionar maior organização, reutilização de componentes e geração automática de evidências de execução.

%% - - - - - - - - - - - - - - - - - - - - - - - - - - - - - - - - - - -
\section{ Descrição do Projeto}
\label{sec:opcoes}


\par Para a execução dos testes de interface, o projeto faz uso da \textit{SeleniumLibrary}\footnote{\url{SeleniumLibrary}}, que integra o \textit{Robot Framework} ao \textit{Selenium WebDriver}\footnote{\url{https://www.selenium.dev/documentation/webdriver/}}. Essa combinação permite simular o comportamento real de um usuário no navegador. O \textit{Selenium} é um dos \textit{frameworks} mais consolidados para automação de navegadores, oferecendo suporte a diferentes navegadores e sistemas operacionais, o que garante flexibilidade e maior realismo nos cenários de teste.

\par Além do foco na automação de testes, o trabalho busca através da integração com a inteligencia artificial (IA) com a \textit{api} do \textit{gemini}\footnote{\url{https://ai.google.dev/gemini-api/docs}}, gerar um ambiente muito eficiente onde as falhas encontradas durante a execução dos testes não apenas são detectadas, mas também analisadas automaticamente. A IA sugere hipóteses sobre as possíveis causas e até caminhos para uma depuração mais rápida, servindo como um apoio extra para as equipes de \textit{quality assurance} (QA) e de desenvolvimento.

\par Outro ponto importante é a conexão com o \textit{GitHub Issues}\footnote{\url{https://github.com/features/issues}}. Sempre que um defeito é identificado e diagnosticado, ele pode ser registrado automaticamente em um repositório de versionamento. Isso torna o fluxo de trabalho mais ágil e colaborativo, reduzindo atividades manuais e garantindo a rastreabilidade entre o defeito, sua análise e o acompanhamento da correção.

\par Sendo assim, o objetivo deste estudo de caso vai além de apenas validar a execução de casos de teste. Ele busca mostrar o potencial da automação inteligente, unindo o \textit{Robot Framework}, o \textit{Selenium} e recursos de IA para apoiar diagnósticos e tornar o desenvolvimento de software ainda mais orientado à qualidade.

%% - - - - - - - - - - - - - - - - - - - - - - - - - - - - - - - - - - -
\section{Contexto da Aplicação de Testes}
\label{sec:contaplitest}

\par Os testes automatizados desenvolvidos neste projeto foram aplicados sobre uma aplicação web pública e amplamente utilizada em treinamentos de automação: o portal \textit{The Internet Herokuapp}\footnote{https://the-internet.herokuapp.com}. Essa plataforma foi escolhida por oferecer diversos cenários controlados de teste, como formulários de login, \textit{upload} de arquivos, botões dinâmicos e elementos interativos, que permitem validar diferentes aspectos do comportamento de uma aplicação \textit{web}.
\par Durante a execução, o \textit{Robot Framework}, em conjunto com a \textit{SeleniumLibrary}, foi responsável por controlar o navegador e reproduzir as ações do usuário, como preencher campos, clicar em botões e validar mensagens de erro. Esse processo possibilitou avaliar a estabilidade da aplicação e a capacidade do \textit{framework} em detectar e registrar falhas de interface de forma confiável.
\par Além disso, o contexto do teste foi ampliado com a integração de ferramentas complementares. A \textit{API Gemini} foi utilizada para interpretar automaticamente as falhas registradas, fornecendo análises e hipóteses sobre suas possíveis causas, enquanto a integração com o \textit{GitHub Issues} garantiu que cada erro detectado fosse documentado diretamente em um repositório de versionamento, permitindo o acompanhamento e rastreabilidade do ciclo de correção.
\par Dessa forma, o contexto da aplicação de testes reflete um ambiente que simula situações reais de uso e falhas comuns em sistemas web, demonstrando como a automação aliada à inteligência artificial pode contribuir para tornar o processo de validação mais inteligente, ágil e integrado ao fluxo de desenvolvimento de software.

%% - - - - - - - - - - - - - - - - - - - - - - - - - - - - - - - - - - -
\section{Arquitetura e Estrutura do Projeto}
\label{sec:arqestproj}
\par A arquitetura do projeto foi concebida para integrar, de forma contínua e automatizada, diferentes componentes responsáveis pela execução dos testes funcionais, análise inteligente de falhas e abertura de \textit{issues} no repositório de desenvolvimento. Essa integração ocorre dentro de um fluxo de \textit{CI/CD} (Integração Contínua/Entrega Contínua), garantindo que cada nova alteração no código seja validada, analisada e rastreada.

\par A Figura~\ref{fig:bpmn} apresenta uma visão geral do processo, modelado em notação \textit{BPMN} (Business Process Model and Notation). Esse diagrama ilustra o ciclo completo de automação, desde o disparo dos testes até a abertura automática de uma \textit{issue} em caso de falha.

\begin{figure}[H]
	\centering
	\includegraphics[width=0.7\linewidth]{fig/BPMN}
	\caption{Fluxo BPMN da automação proposta}
	\label{fig:bpmn}
\end{figure}

\par O fluxo inicia-se quando o repositório recebe um novo \textit{commit} do usuário, acionando automaticamente a rotina de testes. Além disso, a pipeline também é executada em horários pré-determinados, permitindo verificações periódicas mesmo sem alterações recentes no código. A suíte de testes, implementada em \textit{Robot Framework}, utiliza a biblioteca \textit{SeleniumLibrary} para interação com a interface web, realizando validações funcionais e simulando a rotina de uso do sistema.

\par Em caso de falha, as evidências são capturadas automaticamente (capturas de tela, logs e relatórios detalhados). Esse conjunto de informações é então enviado à \textit{API Gemini}, que analisa o erro, identifica possíveis causas e sugere ações de depuração. Com base nesse diagnóstico, é criada automaticamente uma \textit{issue} no GitHub, contendo título, descrição, evidências e o parecer gerado pela IA.

\par A estrutura modular do projeto, separada em arquivos de teste e arquivos de palavras-chave (\textit{resource files}), permite organização clara e reutilização de comandos. Esse padrão facilita a manutenção, amplia a escalabilidade da suíte de testes e torna o fluxo automatizado mais robusto e confiável.

\par Em síntese, a arquitetura proposta combina ferramentas de automação, inteligência artificial e versionamento de forma integrada, garantindo rastreabilidade completa entre código, testes, falhas, análises e correções.


%% - - - - - - - - - - - - - - - - - - - - - - - - - - - - - - - - - - -
\section{Tecnologias Utilizadas}
\label{sec:tecnol}

\par O desenvolvimento deste projeto envolveu o uso de diversas tecnologias que, em conjunto, permitiram construir um ambiente de automação robusto, integrado e inteligente. Cada uma delas desempenha um papel específico na execução dos testes e na análise automatizada dos resultados.

\subsection{\textit{Robot Framework}}
\label{sub:robot}

\par O \textbf{Robot Framework} é a base do projeto, sendo um \textit{framework} de automação de testes de código aberto amplamente utilizado para testes funcionais, de aceitação e de integração. Seu principal diferencial é o uso de uma linguagem baseada em \textit{keywords}, que torna a escrita dos testes mais legível e acessível, mesmo para profissionais que não possuem conhecimento avançado em programação. Além disso, sua arquitetura modular permite a integração com diferentes bibliotecas e ferramentas externas, o que facilita a expansão das funcionalidades.

\subsection{\textit{Selenium WebDriver} e \textit{SeleniumLibrary}}
\label{sub:selenium}

\par A automação da interface foi implementada por meio da \textbf{\textit{SeleniumLibrary}}, que faz a ponte entre o \textit{Robot Framework} e o \textbf{\textit{Selenium WebDriver}}. Essa integração permite controlar o navegador da mesma forma que um usuário real, simulando ações como cliques, preenchimento de campos e navegação entre páginas. O \textit{Selenium} é uma das tecnologias mais consolidadas para automação de testes em aplicações \textit{web}, oferecendo suporte a múltiplos navegadores e sistemas operacionais, o que garante flexibilidade e portabilidade aos testes.

\subsection{API Gemini}
\label{sub:gemini}

\par A integração com a \textbf{API Gemini}, da Google, representa o elemento de IA do projeto. Sempre que um defeito é detectado durante a execução dos testes, a IA é acionada para interpretar o contexto do defeito e gerar uma análise automática, sugerindo causas técnicas prováveis e ações de depuração. Essa abordagem agrega um nível de automação cognitiva ao processo, reduzindo o tempo de diagnóstico e apoiando as equipes de QA e desenvolvimento na identificação de problemas.

\subsection{GitHub e GitHub Issues}
\label{sub:github}

\par O \textbf{GitHub} foi utilizado tanto como repositório para controle de versão do código quanto como ferramenta de rastreabilidade de falhas. A integração direta com o \textbf{GitHub Issues} permite que cada defeito identificado seja automaticamente registrado como uma ocorrência documentada, contendo detalhes do teste, a mensagem de erro e a análise gerada pela IA. Esse processo automatizado facilita o acompanhamento das correções e promove maior colaboração entre os times envolvidos.
\par Dessa forma, a combinação dessas tecnologias possibilitou a criação de um ecossistema de automação completo, no qual a execução de testes, a detecção de falhas, a análise inteligente e o registro de ocorrências estão integrados em um único fluxo automatizado e eficiente.


%% - - - - - - - - - - - - - - - - - - - - - - - - - - - - - - - - - - -
\section{Fluxo de Execução dos Testes com IA}
\label{sec:fluxoteste}

\par O fluxo de execução dos testes automatizados foi projetado para demonstrar como a integração entre o \textit{Robot Framework}, a IA (API Gemini) e o \textit{GitHub Issues} pode criar um ciclo de validação e análise contínua, tornando o processo de garantia da qualidade mais inteligente e autônomo.

\subsection{Etapa 1: Inicialização do Teste}
\label{sub:etapa1}

\par A execução se inicia no \textit{Robot Framework}, que orquestra todos os componentes do processo. O ambiente de testes é configurado com as bibliotecas necessárias, como \textit{SeleniumLibrary} (para controle do navegador), \textit{RequestsLibrary} (para comunicação HTTP) e \textit{AllureLibrary} (para geração de relatórios visuais). Em seguida, o \textit{framework} abre o navegador definido e acessa a aplicação alvo — neste caso, o portal \textit{The Internet Herokuapp} — para iniciar os cenários de teste funcionais.

\subsection{Etapa 2: Execução das Ações e Validação}
\label{sub:etapa2}

\par Durante os testes, o \textit{Selenium} simula as interações de um usuário real, realizando ações como preencher formulários, submeter dados e validar respostas da interface. Caso a aplicação se comporte conforme o esperado, o teste é marcado como bem-sucedido e registrado no relatório de execução. No entanto, se uma inconsistência for identificada (como um erro de autenticação ou falha de carregamento), o \textit{framework} aciona automaticamente o módulo de análise com IA.

\subsection{Etapa 3: Análise Automática da Falha com IA}
\label{sub:etapa3}

\par Quando uma falha é detectada durante a execução da suíte de testes, o erro correspondente é capturado e enviado à \textit{API Gemini}, da Google. A execução desses testes ocorre de duas formas: automaticamente a cada novo \textit{commit} realizado pelo usuário no repositório e, adicionalmente, em horários pré-determinados definidos na rotina de integração contínua. Dessa forma, cada falha registrada pode ser associada tanto a uma alteração recente no código quanto a verificações periódicas do estado geral do sistema.

\par A \textit{API Gemini} recebe como entrada o contexto completo do evento, incluindo a mensagem de erro retornada, o nome do teste afetado, o \textit{commit} em execução (quando aplicável) e o autor responsável pela modificação, e gera uma resposta textual contendo:

\begin{itemize}
	\item três possíveis causas técnicas para a falha;
	\item duas sugestões de ações de depuração rápida.
\end{itemize}

\par Essa análise automática permite identificar a origem do problema de forma imediata, reduzindo significativamente o tempo necessário para diagnóstico manual e fornecendo insights técnicos diretamente para a equipe de desenvolvimento. Além disso, a combinação entre disparo por \textit{commit} e execuções agendadas aumenta a cobertura, a rastreabilidade e a confiabilidade do processo de validação contínua.

\subsection{Etapa 4: Registro da Falha no GitHub Issues}
\label{sub:regFalha}

\par Após o retorno da análise da IA, o sistema utiliza a \textbf{API do GitHub} para registrar automaticamente uma \textit{issue} no repositório do projeto. Este registro contém:
\begin{itemize}
	\item título descritivo da falha;
	\item mensagem de erro capturada;
	\item diagnóstico gerado pela IA (causas e ações recomendadas);
	\item informações sobre o commit e o responsável pela execução.
\end{itemize}
\par Essa integração garante rastreabilidade entre o erro ocorrido, sua análise e o acompanhamento da correção, permitindo que o fluxo de trabalho entre QA e desenvolvimento se torne mais colaborativo e eficiente.

\subsection{Etapa 5: Relatórios e Evidências}
\label{sub:relato}

\par Por fim, os resultados completos são consolidados pelo \textbf{Allure Report}, que gera relatórios visuais contendo o status de cada teste, as evidências de execução (como capturas de tela) e o histórico das execuções anteriores. Dessa forma, é possível acompanhar a evolução da qualidade da aplicação ao longo do tempo.

\par Em síntese, o fluxo de execução dos testes combina automação funcional, análise inteligente e rastreabilidade contínua, demonstrando como a união entre \textbf{Robot Framework}, \textbf{Selenium}, \textbf{IA Gemini} e \textbf{GitHub Issues} pode criar um ecossistema de validação moderno, ágil e orientado à qualidade.



\chapter{Considerações Finais}
\label{cap:consFin}

\par Este capítulo apresenta os resultados obtidos durante a execução do estudo de caso, destacando o impacto real do uso do Robot Framework integrado à Inteligência Artificial na automação de testes funcionais. Mais do que validar apenas a execução de suites automatizadas, o objetivo deste trabalho foi compreender se a IA pode agir como um agente complementar de análise, diagnóstico e apoio ao fluxo de QA, reduzindo esforço manual, acelerando investigação de falhas e tornando o ciclo de desenvolvimento mais orientado à evidência

\par Ao longo dos testes, avaliamos alguns pontos importantes: a capacidade de detectar erros, a qualidade das análises feitas pela IA, a consistência do rastreamento automático no GitHub Issues e o quanto a ferramenta agregou de valor, seja pelo reaproveitamento de componentes, seja pela facilidade de manter os testes.

\par Com isso, os resultados que serão apresentados aqui mostram não só que o modelo proposto é tecnicamente viável, mas também que ele tem aplicação real, agrega relevância ao processo de garantia de qualidade e tem potencial para evoluir em direção a um contexto de automação inteligente de testes.
%% - - - - - - - - - - - - - - - - - - - - - - - - - - - - - - - - - - -
\section{Resultados Obtidos nos Testes}
\label{sec:resultados}

\par A suíte de automação desenvolvida contemplou tanto os cenários de autenticação quanto funcionalidades adicionais disponibilizadas pelo ambiente de testes \textit{The Internet}, abrangendo componentes como \textit{checkboxes}, \textit{dropdown menus}, \textit{upload} de arquivos, carregamento dinâmico, detecção de imagens quebradas e identificação de erros ortográficos. Ao todo, foram executados oito casos de teste, cobrindo tanto cenários positivos quanto negativos e comportamentos inesperados.

\subsection{Volume total de casos executados}
\par A suíte contemplou 8 testes automatizados, distribuídos em:
\begin{itemize}
	\item Testes de \textit{login} (fluxos válidos e inválidos);
	\item Testes de validação funcional (\textit{checkboxes}, \textit{dropdown}, \textit{dynamic loading});
	\item Testes de robustez (\textit{upload}, \textit{typos}, imagens quebradas).
\end{itemize}

\par Esse conjunto permitiu validar tanto o funcionamento essencial dos componentes quanto cenários extremos ou propositalmente defeituosos, disponíveis no ambiente de testes.


\subsection{Execução com sucesso e falhas detectadas}
\par A execução completa gerou:
\begin{itemize}
	\item 5 testes bem-sucedidos;
	\item 3 testes com falhas detectadas automaticamente.
\end{itemize}

\par Algumas falhas identificadas não foram geradas pela automação, mas representam erros reais do ambiente, como:
\begin{itemize}
	\item imagens quebradas na rota \texttt{/broken\_images};
	\item variações textuais e erros ortográficos em \texttt{/typos};
	\item possíveis perdas de estado no carregamento dinâmico.
\end{itemize}

\subsection{Tempo médio e ganho proporcionado pela IA}
\par A suíte apresentou um tempo médio de execução de aproximadamente 1 minuto, com tempo individual de execução variando de acordo com a complexidade de cada cenário. A integração com a IA reduziu significativamente o esforço de análise das falhas, uma vez que o modelo Gemini foi capaz de gerar diagnósticos técnicos em poucos segundos.
  
  \begin{figure}[H]
  	\centering
  	\includegraphics[width=0.7\linewidth]{fig/Relatorio}
    \caption{Relatório gerado automaticamente pelo Robot Framework contendo o resumo da execução dos testes.}
  	\label{fig:relatorio}
  \end{figure}
  
  
\subsection{Evidências de execução}
\par Durante as execuções, foram coletadas evidências como:
\begin{itemize}
	\item capturas de tela automáticas nos pontos de falha;
	\item logs detalhados gerados pelo Robot Framework;
	\item relatórios HTML (\texttt{log.html} e \texttt{report.html});
	\item registros das análises feitas pela IA;
	\item issues geradas automaticamente no GitHub.
\end{itemize}


\section{Análise da Contribuição da IA no Diagnóstico das Falhas}
\label{sec:contribuicaoIA}

\par A integração da Inteligência Artificial no processo de automação trouxe benefícios relevantes para a triagem e diagnóstico de falhas, reduzindo o tempo de análise humana e evitando retrabalho. O modelo Gemini se mostrou eficiente ao interpretar erros, propor causas possíveis e sugerir ações corretivas com base na execução do teste.

\subsection{Impacto real na identificação das falhas}
\par A IA foi capaz de analisar mensagens de erro capturadas pela automação, correlacionar dados do ambiente de execução e gerar diagnósticos consistentes. Isso permitiu acelerar significativamente o processo de detecção de falhas e gerar documentação clara para suporte e desenvolvimento.

\subsection{Exemplos reais de falhas analisadas pela IA}
\par Entre os exemplos observados, destacam-se:
\begin{itemize}
	\item imagens quebradas, onde a IA sugeriu revisão dos caminhos de recursos estáticos;
	\item erro ortográfico na página \texttt{/typos}, interpretado corretamente como parte do comportamento do ambiente de testes;
	\item inconsistência no carregamento do \textit{dynamic loading}, onde a IA recomendou revisão de timeout e inspeção do DOM;
\end{itemize}

\subsection{Comparação com abordagem manual}
\par Antes da integração da IA, cada falha exigia análise manual, como inspeção de logs, reprodução do teste e abertura manual de issues. Com a IA, esse processo tornou-se automático, reduzindo esforço humano e tempo de resposta.

\par Dessa forma, observou-se uma redução significativa no ciclo de detecção--documentação--correção, aumentando a eficiência do fluxo entre QA e desenvolvimento.

\subsection{Limitações observadas}
\par Apesar da eficácia, algumas limitações foram identificadas:
\begin{itemize}
	\item dependência da clareza da mensagem de erro extraída dos testes;
	\item diagnósticos menos precisos em falhas muito técnicas ou contextos complexos;
	\item necessidade de interpretação humana final antes de decisões críticas.
\end{itemize}


\section{Integração com GitHub Issues e Rastreabilidade}
\label{sec:github}

\par A integração direta entre a automação, a IA e o GitHub permitiu que cada falha identificada fosse transformada automaticamente em uma issue. Este fluxo garantiu maior rastreabilidade, melhor organização do backlog e comunicação mais eficiente entre testers e desenvolvedores.


\begin{figure}[H]
	\centering
	\includegraphics[width=0.75\linewidth]{fig/IssueGerada}
	\caption{Issue criada automaticamente no GitHub contendo o diagnóstico gerado pela API Gemini.}
	\label{fig:issuegerada}
\end{figure}


\subsection{Agilidade no fluxo de QA e Desenvolvimento}
\par O processo reduziu a latência entre a ocorrência do erro e sua formalização, uma vez que:
\begin{itemize}
	\item a detecção da falha ocorre no teste;
	\item a IA analisa a falha imediatamente;
	\item a issue é criada automaticamente com título, corpo, labels e diagnóstico técnico.
\end{itemize}

\subsection{Número de issues geradas}
\par Durante as execuções da suíte, foram geradas:
\begin{itemize}
	\item 3 issues referentes a erros reais;
	\item 1 issues provenientes de testes negativos intencionais.
\end{itemize}

\par Isso evidenciou a capacidade da suíte em fornecer documentação automatizada para cada falha detectada.

\subsection{Benefícios diretos para o time}
\par Entre os benefícios concretos, destacam-se:
\begin{itemize}
	\item maior rastreabilidade entre falha, evidência, diagnóstico e correção;
	\item histórico organizado para auditoria e melhoria contínua;
	\item eliminação do trabalho manual de documentação;
	\item agilidade no fluxo de manutenção.
\end{itemize}




\section{Discussão sobre o Uso do Robot Framework para Automação Funcional}
\label{sec:robotframework}

\par O Robot Framework demonstrou ser uma ferramenta adequada para automação funcional devido à sua sintaxe simples, modularidade e grande ecossistema de bibliotecas.

\subsection{Pontos fortes}
\par Entre as características positivas, destacam-se:
\begin{itemize}
	\item sintaxe altamente legível;
	\item separação clara entre lógica, keywords e testes;
	\item suporte nativo a logs e relatórios automáticos;
	\item integração fácil com Selenium, Requests e APIs externas;
	\item compatibilidade com CI/CD.
\end{itemize}

\subsection{Pontos fracos}
\par Algumas limitações também foram identificadas:
\begin{itemize}
	\item forte dependência de espaçamento e formatação;
	\item debugging limitado em comparação com linguagens tradicionais;
	\item dependência da estabilidade do Selenium WebDriver;
	\item necessidade de maior controle em execuções headless.
\end{itemize}

\subsection{Benefícios reais para times e escala}
\par O uso do Robot Framework possibilitou:
\begin{itemize}
	\item aumento da produtividade no desenvolvimento de testes;
	\item redução da curva de aprendizado para novos integrantes do time;
	\item manutenção facilitada pela modularidade das keywords;
	\item escalabilidade do conjunto de testes;
	\item integração fluida com processos de CI/CD e ferramentas de rastreabilidade.
\end{itemize}



%<<<<<<< Updated upstream
%<<<<<<< Updated upstream

%=======
%>>>>>>> Stashed changes
%=======
%>>>>>>> Stashed changes

%------------------------------------------------------------ BIBLIOGRAFIA %
\cleardoublepage
%\nocite{*} %%% Retire esta linha para gerar a bibliografia com apenas as
           %%% referências usadas no seu texto!
\arial
%\bibliographystyle{abnt}
%<<<<<<< Updated upstream
%<<<<<<< Updated upstream
\bibliography{../tcc2_gabriel_porto/bib/tcc2-gporto} %%% Nomes dos seus arquivos .bib
%=======
%\bibliography{./bib/tcc2-gporto} %%% Nomes dos seus arquivos .bib
%>>>>>>> Stashed changes
%=======
%\bibliography{./bib/tcc2-gporto} %%% Nomes dos seus arquivos .bib
%>>>>>>> Stashed changes
\label{ref-bib}

%--------------------------------------------------------------- APÊNDICES %
\apendices

%\input{./pos/apend_I}
%\input{./pos/apend_II}

\end{document}

%------------------------------------------------------------------------- %
%        F I M   D O  A R Q U I V O :  m o d e l o - t e s e . t e x       %
%------------------------------------------------------------------------- %
