\left( \documentclass[12pt]{article}
\usepackage[utf8]{inputenc}
\usepackage[brazil]{babel}
\usepackage{geometry}
\usepackage{graphicx}
\usepackage{longtable}
\geometry{margin=2.5cm}

\title{Documentação: Processo de Validação Automatizada com Robot Framework e Gemini AI}
\author{Gabril Porto do Carmo}

\begin{document}

\maketitle

\section{Visão Geral}

Este documento descreve um processo de validação automatizada que realiza testes de login sempre que há alterações no código ou em horários definidos. Em caso de falha, o erro é analisado com inteligência artificial (Gemini) e, caso necessário, uma issue é criada automaticamente no GitHub com informações detalhadas para correção.

\section{Componentes do Sistema}

\begin{longtable}{|p{5cm}|p{9cm}|}
\hline
\textbf{Componente} & \textbf{Função} \\
\hline
Robot Framework & Executa os testes automatizados com Selenium. \\
\hline
Gemini API & Analisa erros e sugere diagnósticos técnicos com IA. \\
\hline
GitHub Actions & Orquestra a execução automatizada dos testes. \\
\hline
GitHub Issues & Centraliza os relatórios de erro com diagnóstico e contexto. \\
\hline
\end{longtable}

\section{Fluxo Detalhado}

\subsection{Gatilhos (Triggers)}

Os testes automatizados podem ser disparados de duas formas:
\begin{itemize}
    \item Após cada push ou pull request para o branch principal (\texttt{main/master}).
    \item Duas vezes ao dia, em horários previamente definidos (agendamento via \texttt{cron}).
\end{itemize}

\subsection{Execução dos Testes}

O GitHub Actions cria um ambiente Ubuntu, instala Python e as bibliotecas necessárias (Robot Framework, Selenium, Requests), e executa o teste de login. O teste roda em modo \textit{headless} para ambientes CI.

\subsection{Tratamento de Falhas}

Caso o teste falhe com um erro 500 ou mensagem semelhante, o sistema captura a tela, envia o erro para análise via Gemini AI, e cria uma issue no GitHub com o diagnóstico gerado.

\subsection{Criação da Issue}

A issue é criada automaticamente no GitHub contendo:
\begin{itemize}
    \item Horário da falha
    \item Link para o commit causador
    \item Screenshot do erro
    \item Diagnóstico do Gemini com causas e ações sugeridas
    \item Labels: \texttt{bug}, \texttt{automatizado}, \texttt{pós-commit}
\end{itemize}

\section{Exemplo de Saída}

\subsection{Issue no GitHub}

Exemplo de issue criada com diagnóstico do Gemini e imagem do erro.

\subsection{Logs do GitHub Actions}

Exemplo de saída esperada nos logs:

\begin{verbatim}
2024-05-27 10:00:00 | ERRO: 500 Internal Server Error
2024-05-27 10:00:05 | Gemini: Causa 1 - Banco de dados offline
2024-05-27 10:00:10 | Issue criada: #123
\end{verbatim}

\end{document}
