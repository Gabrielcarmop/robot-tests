%\chapter{Introdu\c{c}\~ao - Tema e Problematiza\c{c}\~ao}
\chapter{Introdução}
\label{cap:Intro}

\par Os testes de software são essenciais para a garantia e validação das funcionalidades de um projeto, desde a concepção inicial até sua versão final. Eles terão como objetivo validar funcionalidades, identificar falhas e assegurar a qualidade do software, evitando que erros cheguem ao usuário.
\par No âmbito dos testes, existirão diversos tipos e técnicas que poderão ser aplicados de acordo com as funcionalidades e a estrutura do projeto. Os testes poderão ser manuais ou automatizados, abrangendo desde testes unitários, que validarão individualmente uma parte do código, até testes de integração e de sistema, que avaliarão a interação entre módulos e o funcionamento completo do software. Cada abordagem terá seu papel na construção de um processo confiável de desenvolvimento e entrega ao usuário.
\par Em alguns casos, os testes tornam-se ainda mais indispensáveis, especialmente quando envolvem funcionalidades críticas que podem comprometer completamente o software e causar falhas catastróficas no mundo real. Nesse cenário, a validação contínua da qualidade e da segurança torna-se uma exigência constante. Para isso, os processos e as técnicas de teste devem ser cuidadosamente planejados, priorizando as melhores estratégias para cobrir todas as funcionalidades essenciais do software.
\par Durante o desenvolvimento e a manutenção do software, mudanças e melhorias serão necessárias, e os testes deverão acompanhar essas alterações. Uma das alternativas para esse monitoramento e aperfeiçoamento será a aplicação de testes automatizados, que surgirão como uma solução eficaz para manter o processo de validação sempre atualizado.
\par Para que essa estratégia seja realmente eficaz, os testes não poderão ser conduzidos como uma etapa isolada ou final do desenvolvimento. Eles precisarão ser planejados desde o início do projeto e integrados ao fluxo de trabalho da equipe, tornando-se parte natural e contínua do ciclo de vida do software. Esse paradigma é fortemente defendido por metodologias ágeis e pela cultura DevOps, que promoverão práticas como Integração Contínua (CI) e Entrega Contínua (CD), estimulando que a validação da qualidade ocorra em todas as fases do desenvolvimento, de forma automatizada e iterativa.
\par Nesse contexto, formula-se a questão que guiará o desenvolvimento deste trabalho:
de que maneira a integração entre automação de testes e mecanismos de inteligência artificial poderá aprimorar o processo de análise de falhas, ampliar a rastreabilidade e tornar o ciclo de garantia da qualidade mais eficiente?
\par Assim, insere-se neste cenário a proposta deste trabalho, que explorará a automação de testes em conjunto com mecanismos modernos de inteligência artificial e pipelines de CI/CD, demonstrando como essas abordagens poderão elevar a confiabilidade, a rastreabilidade e a eficiência do processo de desenvolvimento de software.

\section{Justificativa} 
\label{sec:justif}
\par   Este trabalho surgiu do interesse pessoal do autor, pela área de testes de software e pela busca por soluções que automatizem estes testes. A validação automatizada exige uma abordagem constante e personalizada. O grande desafio é garantir que todas as funcionalidades do software continuem funcionando corretamente, especialmente após alterações no seu código-fonte, o que é fundamental para garantir a qualidade e a segurança do software.
\par Com foco principal na melhoria dos testes de software por meio da automação, este trabalho também utiliza Inteligência Artificial (IA) para validar e apoiar a tomada de decisões. Isso não apenas reduz o risco de problemas não detectados, mas também contribui para um ciclo de desenvolvimento mais ágil e seguro.
\par   Além disso, este trabalho também busca fomentar discussões sobre como a automação de testes e o uso de IA podem melhorar a validação de sistemas de software. Ao detectar defeitos, o quanto antes, agiliza o ciclo de desenvolvimento, beneficiando os envolvidos no processo de desenvolvimento de software.
\par  Em resumo, a ideia é o  aprimoramento contínuo e a ampliação do conhecimento sobre todos os aspectos relacionados à automação de testes, beneficiando assim a comunidade de profissionais e entusiastas da área.

\section{Objetivos}
\label{sec:objtivos}

\subsection{Objetivo Geral}
\label{sub:objGeral}

\par    Esse estudo tem como objetivo analisar como a automação de teste, combinado com o uso de inteligencia artificial podem melhorar o teste de software, com foco na validação contínua das funcionalidades do software, promovendo maior qualidade, segurança e eficiência no ciclo de desenvolvimento.

\subsection{Objetivos Específicos}
\label{sub:objEspec}
\begin{itemize}
    \item   Compreender como a automação de testes pode ser aplicada de forma eficaz em todas as partes de um  software;
    \item   Analisar como a Inteligência Artificial pode ser utilizada para identificar falhas e auxiliar na tomada de decisões durante os ciclos de testes;
    \item   Apresentar casos práticos onde o uso do \textit{Robot Framework}\footnote{https://robotframework.org/} e do \textit{GitHub Actions}\footnote{https://github.com/features/actions} contribui para a detecção de defeitos e a integração contínua do código;
    \item   Avaliar a eficácia da validação automatizada.
\end{itemize}

\section{Metodologia}
\par Este trabalho se caracteriza como uma pesquisa aplicada, de natureza qualitativa e exploratória. Sua abordagem metodológica concentra-se na investigação e experimentação prática de técnicas voltadas à detecção e análise de falhas em sistemas web, utilizando ferramentas de automação de testes e plataformas de versionamento de código como elementos centrais do estudo.
\par Na fase inicial, realizou-se um levantamento teórico sobre conceitos fundamentais de testes de software, automação e qualidade, apoiado em obras clássicas e materiais contemporâneos da área, incluindo Myers \cite{myers2011}, Pressman \cite{pressman2002}, McCabe \cite{mccabe1976}, bem como artigos, guias técnicos e documentações oficiais relacionadas ao Robot Framework, Selenium WebDriver, integração contínua e técnicas funcionais e estruturais de teste. Esse levantamento teve como finalidade fornecer a base conceitual necessária para orientar o desenvolvimento do experimento.
\par  Em seguida, foram desenvolvidos testes automatizados utilizando Robot Framework e \textit{Selenium WebDriver}. Os primeiros testes foram aplicados em um site com cenários de erro proposital\footnote{Disponível em: \href{https://the-internet.herokuapp.com/login}{https://the-internet.herokuapp.com/login}. Acesso em: 22 jun. 2025.}. Esse ambiente foi ideal para validar o funcionamento da estratégia de testes automatizados.
\par  Depois dessa fase inicial, o projeto será implementada em um sistema real para testar sua eficácia em um ambiente de produção. Os testes foram configurados para rodar automaticamente sempre que um novo \textit{commit} é feito no repositório do projeto no GitHub, ou em horários definidos, com o uso do \textit{GitHub Actions}. Quando uma falha é detectada, o sistema gera automaticamente uma \textit{issue} no próprio \textit{GitHub}, contendo \textit{prints} da tela, \textit{logs} e uma descrição resumida com a ajuda de uma inteligência artificial que interpreta o erro.
\par    Por fim,  os resultados são analisados, observando os tipos de erros mais frequentes, o tempo ate encontrar  a falha, e a utilidade das informações geradas para os desenvolvedores.


\section{Organização do Trabalho}
\label{sec:organ}

Para facilitar o alcance dos objetivos especificados, além desta Introdução, o restante deste trabalho está organzado da seguinte forma:

\begin{itemize}
	\item No Capítulo \ref{cap:fundTeo} é apresentada a fundamentação teórica, que embasa todo este trabalho.
	\item No Capítulo \ref{cap:testAut} é apresentado o projeto propriamente dito...
	\item No Capítulo \ref{cap:consFin} são apresentadas as considerações finais e os possíveis desdobramentos futuros deste trabalho.
\end{itemize}


