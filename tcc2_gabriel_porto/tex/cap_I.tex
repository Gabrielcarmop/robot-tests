%\chapter{Introdu\c{c}\~ao - Tema e Problematiza\c{c}\~ao}
\chapter{Introdução}
\label{cap:Intro}

\par Os testes de software são essenciais para a garantia e validação das funcionalidades de um projeto, desde a concepção inicial até sua versão final. Eles têm como objetivo validar funcionalidades, identificar falhas e assegurar a qualidade do software, evitando que erros cheguem ao usuário.
\par No âmbito dos testes, existem diversos tipos e técnicas que podem ser aplicados de acordo com as funcionalidades e a estrutura do projeto. Os testes podem ser manuais ou automatizados, abrangendo desde testes unitários, que validam individualmente uma parte do código, até testes de integração e de sistema, que avaliam a interação entre módulos e o funcionamento completo do software. Cada abordagem tem seu papel na construção de um processo confiável de desenvolvimento e entrega ao usuário.
\par Em alguns casos, os testes tornam-se ainda mais indispensáveis, especialmente quando envolvem funcionalidades críticas que podem comprometer completamente o software e causar falhas catastróficas no mundo real. Nesse cenário, a validação contínua da qualidade e da segurança torna-se uma exigência constante. Para isso, os processos e as técnicas de teste devem ser cuidadosamente planejados, priorizando as melhores estratégias para cobrir todas as funcionalidades essenciais do software.
\par Durante o desenvolvimento e a manutenção do software, mudanças e melhorias são necessárias, e os testes devem acompanhar essas alterações. Uma das alternativas para esse monitoramento e aperfeiçoamento é a aplicação de testes automatizados, que surgem como uma solução eficaz para manter os testes sempre atualizados.
\par Porém, para que essa técnica seja realmente eficaz, os testes não podem ser uma etapa isolada do desenvolvimento; eles devem estar presentes desde o início. É essencial que façam parte de cada momento do ciclo de vida do software, tornando-se parte natural do trabalho da equipe. Essa abordagem tem sido cada vez mais adotada em metodologias ágeis e na cultura DevOps, promovendo uma integração contínua da qualidade ao longo de todo o processo de desenvolvimento.


\textcolor{red}{[é necessário enriquecer um pouco mais esta introdução]}

\section{Justificativa} 
\label{sec:justif}
\par   Este trabalho surgiu do interesse pessoal do autor, pela área de testes de software e pela busca por soluções que automatizem estes testes. A validação automatizada exige uma abordagem constante e personalizada. O grande desafio é garantir que todas as funcionalidades do software continuem funcionando corretamente, especialmente após alterações no seu código-fonte, o que é fundamental para garantir a qualidade e a segurança do software.
\par Com foco principal na melhoria dos testes de software por meio da automação, este trabalho também utiliza Inteligência Artificial (IA) para validar e apoiar a tomada de decisões. Isso não apenas reduz o risco de problemas não detectados, mas também contribui para um ciclo de desenvolvimento mais ágil e seguro.
\par   Além disso, este trabalho também busca fomentar discussões sobre como a automação de testes e o uso de IA podem melhorar a validação de sistemas de software. Ao detectar defeitos, o quanto antes, agiliza o ciclo de desenvolvimento, beneficiando os envolvidos no processo de desenvolvimento de software.
\par  Em resumo, a ideia é o  aprimoramento contínuo e a ampliação do conhecimento sobre todos os aspectos relacionados à automação de testes, beneficiando assim a comunidade de profissionais e entusiastas da área.

\section{Objetivos}
\label{sec:objtivos}

\subsection{Objetivo Geral}
\label{sub:objGeral}

\par    Esse estudo tem como objetivo analisar como a automação de teste, combinado com o uso de inteligencia artificial podem melhorar o teste de software, com foco na validação contínua das funcionalidades do software, promovendo maior qualidade, segurança e eficiência no ciclo de desenvolvimento.

\subsection{Objetivos Específicos}
\label{sub:objEspec}
\begin{itemize}
    \item   Compreender como a automação de testes pode ser aplicada de forma eficaz em todas as partes de um  software;
    \item   Analisar como a Inteligência Artificial pode ser utilizada para identificar falhas e auxiliar na tomada de decisões durante os ciclos de testes;
    \item   Apresentar casos práticos onde o uso do \textit{Robot Framework}\footnote{https://robotframework.org/} e do \textit{GitHub Actions} \footnote{https://github.com/features/actions} contribui para a detecção de defeitos e a integração contínua do código;
    \item   Avaliar a eficácia da validação automatizada.
\end{itemize}

\section{Metodologia}
\par    Este trabalho se caracteriza como uma pesquisa aplicada, de natureza qualitativa e exploratória. O principal objetivo \textcolor{red}{[tem que tirar este objetivo daqui, pois tem seção específica para os objetivos]} é avaliar uma abordagem para detectar e propor soluções para falhas em sistemas web, utilizando ferramentas de automação de testes e plataformas de versionamento de código
\par  Na  fase inicial da pesquisa, realizou-se uma revisão bibliográfica abrangente, com análise das publicações de \cite{myers2011} sobre testes de software, complementadas por livros, artigos e materiais técnicos atuais sobre testes automatizados
\par  Em seguida, foram desenvolvidos testes automatizados utilizando Robot Framework e \textit{Selenium WebDriver}. Os primeiros testes foram aplicados em um site com cenários de erro proposital\footnote{Disponível em: \href{https://the-internet.herokuapp.com/login}{https://the-internet.herokuapp.com/login}. Acesso em: 22 jun. 2025.}. Esse ambiente foi ideal para validar o funcionamento da estratégia de testes automatizados.
\par  Depois dessa fase inicial, o projeto será implementada em um sistema real para testar sua eficácia em um ambiente de produção. Os testes foram configurados para rodar automaticamente sempre que um novo \textit{commit} é feito no repositório do projeto no GitHub, ou em horários definidos, com o uso do \textit{GitHub Actions}. Quando uma falha é detectada, o sistema gera automaticamente uma \textit{issue} no próprio \textit{GitHub}, contendo \textit{prints} da tela, \textit{logs} e uma descrição resumida com a ajuda de uma inteligência artificial que interpreta o erro.
\par    Por fim,  os resultados são analisados, observando os tipos de erros mais frequentes, o tempo ate encontrar  a falha, e a utilidade das informações geradas para os desenvolvedores.


\section{Organização do Trabalho}
\label{sec:organ}

Para facilitar o alcance dos objetivos especificados, o restante deste trabalho está organzado da seguinte forma:

\begin{itemize}
	\item No Capítulo \ref{cap:fundTeo} é apresentada a fundamentação teórica, que embasa todo este trabalho.
	\item No Capítulo \ref{cap:testAut} é apresentado o projeto propriamente dito...
	\item No Capítulo \ref{cap:consFin} são apresentadas as considerações finais e os possíveis desdobramentos futuros deste trabalho.
\end{itemize}


