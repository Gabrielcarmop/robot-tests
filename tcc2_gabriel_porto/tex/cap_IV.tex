\chapter{Considerações Finais}
\label{cap:consFin}

\par Este capítulo apresenta os resultados obtidos durante a execução do estudo de caso, destacando o impacto real do uso do Robot Framework integrado à Inteligência Artificial na automação de testes funcionais. Mais do que validar apenas a execução de suites automatizadas, o objetivo deste trabalho foi compreender se a IA pode agir como um agente complementar de análise, diagnóstico e apoio ao fluxo de QA, reduzindo esforço manual, acelerando investigação de falhas e tornando o ciclo de desenvolvimento mais orientado à evidência

\par Ao longo dos testes, avaliamos alguns pontos importantes: a capacidade de detectar erros, a qualidade das análises feitas pela IA, a consistência do rastreamento automático no GitHub Issues e o quanto a ferramenta agregou de valor, seja pelo reaproveitamento de componentes, seja pela facilidade de manter os testes.

\par Com isso, os resultados que serão apresentados aqui mostram não só que o modelo proposto é tecnicamente viável, mas também que ele tem aplicação real, agrega relevância ao processo de garantia de qualidade e tem potencial para evoluir em direção a um contexto de automação inteligente de testes.
%% - - - - - - - - - - - - - - - - - - - - - - - - - - - - - - - - - - -
\section{Resultados Obtidos nos Testes}
\label{sec:resultados}

\par A suíte de automação desenvolvida contemplou tanto os cenários de autenticação quanto funcionalidades adicionais disponibilizadas pelo ambiente de testes \textit{The Internet}, abrangendo componentes como \textit{checkboxes}, \textit{dropdown menus}, \textit{upload} de arquivos, carregamento dinâmico, detecção de imagens quebradas e identificação de erros ortográficos. Ao todo, foram executados oito casos de teste, cobrindo tanto cenários positivos quanto negativos e comportamentos inesperados.

\subsection{Volume total de casos executados}
\par A suíte contemplou 8 testes automatizados, distribuídos em:
\begin{itemize}
	\item Testes de \textit{login} (fluxos válidos e inválidos);
	\item Testes de validação funcional (\textit{checkboxes}, \textit{dropdown}, \textit{dynamic loading});
	\item Testes de robustez (\textit{upload}, \textit{typos}, imagens quebradas).
\end{itemize}

\par Esse conjunto permitiu validar tanto o funcionamento essencial dos componentes quanto cenários extremos ou propositalmente defeituosos, disponíveis no ambiente de testes.


\subsection{Execução com sucesso e falhas detectadas}
\par A execução completa gerou:
\begin{itemize}
	\item 5 testes bem-sucedidos;
	\item 3 testes com falhas detectadas automaticamente.
\end{itemize}

\par Algumas falhas identificadas não foram geradas pela automação, mas representam erros reais do ambiente, como:
\begin{itemize}
	\item imagens quebradas na rota \texttt{/broken\_images};
	\item variações textuais e erros ortográficos em \texttt{/typos};
	\item possíveis perdas de estado no carregamento dinâmico.
\end{itemize}

\subsection{Tempo médio e ganho proporcionado pela IA}
\par A suíte apresentou um tempo médio de execução de aproximadamente 1 minuto, com tempo individual de execução variando de acordo com a complexidade de cada cenário. A integração com a IA reduziu significativamente o esforço de análise das falhas, uma vez que o modelo Gemini foi capaz de gerar diagnósticos técnicos em poucos segundos.
  
  \begin{figure}[H]
  	\centering
  	\includegraphics[width=0.7\linewidth]{fig/Relatorio}
    \caption{Relatório gerado automaticamente pelo Robot Framework contendo o resumo da execução dos testes.}
  	\label{fig:relatorio}
  \end{figure}
  
  
\subsection{Evidências de execução}
\par Durante as execuções, foram coletadas evidências como:
\begin{itemize}
	\item capturas de tela automáticas nos pontos de falha;
	\item logs detalhados gerados pelo Robot Framework;
	\item relatórios HTML (\texttt{log.html} e \texttt{report.html});
	\item registros das análises feitas pela IA;
	\item issues geradas automaticamente no GitHub.
\end{itemize}


\section{Análise da Contribuição da IA no Diagnóstico das Falhas}
\label{sec:contribuicaoIA}

\par A integração da Inteligência Artificial no processo de automação trouxe benefícios relevantes para a triagem e diagnóstico de falhas, reduzindo o tempo de análise humana e evitando retrabalho. O modelo Gemini se mostrou eficiente ao interpretar erros, propor causas possíveis e sugerir ações corretivas com base na execução do teste.

\subsection{Impacto real na identificação das falhas}
\par A IA foi capaz de analisar mensagens de erro capturadas pela automação, correlacionar dados do ambiente de execução e gerar diagnósticos consistentes. Isso permitiu acelerar significativamente o processo de detecção de falhas e gerar documentação clara para suporte e desenvolvimento.

\subsection{Exemplos reais de falhas analisadas pela IA}
\par Entre os exemplos observados, destacam-se:
\begin{itemize}
	\item imagens quebradas, onde a IA sugeriu revisão dos caminhos de recursos estáticos;
	\item erro ortográfico na página \texttt{/typos}, interpretado corretamente como parte do comportamento do ambiente de testes;
	\item inconsistência no carregamento do \textit{dynamic loading}, onde a IA recomendou revisão de timeout e inspeção do DOM;
\end{itemize}

\subsection{Comparação com abordagem manual}
\par Antes da integração da IA, cada falha exigia análise manual, como inspeção de logs, reprodução do teste e abertura manual de issues. Com a IA, esse processo tornou-se automático, reduzindo esforço humano e tempo de resposta.

\par Dessa forma, observou-se uma redução significativa no ciclo de detecção--documentação--correção, aumentando a eficiência do fluxo entre QA e desenvolvimento.

\subsection{Limitações observadas}
\par Apesar da eficácia, algumas limitações foram identificadas:
\begin{itemize}
	\item dependência da clareza da mensagem de erro extraída dos testes;
	\item diagnósticos menos precisos em falhas muito técnicas ou contextos complexos;
	\item necessidade de interpretação humana final antes de decisões críticas.
\end{itemize}


\section{Integração com GitHub Issues e Rastreabilidade}
\label{sec:github}

\par A integração direta entre a automação, a IA e o GitHub permitiu que cada falha identificada fosse transformada automaticamente em uma issue. Este fluxo garantiu maior rastreabilidade, melhor organização do backlog e comunicação mais eficiente entre testers e desenvolvedores.


\begin{figure}[H]
	\centering
	\includegraphics[width=0.75\linewidth]{fig/IssueGerada}
	\caption{Issue criada automaticamente no GitHub contendo o diagnóstico gerado pela API Gemini.}
	\label{fig:issuegerada}
\end{figure}


\subsection{Agilidade no fluxo de QA e Desenvolvimento}
\par O processo reduziu a latência entre a ocorrência do erro e sua formalização, uma vez que:
\begin{itemize}
	\item a detecção da falha ocorre no teste;
	\item a IA analisa a falha imediatamente;
	\item a issue é criada automaticamente com título, corpo, labels e diagnóstico técnico.
\end{itemize}

\subsection{Número de issues geradas}
\par Durante as execuções da suíte, foram geradas:
\begin{itemize}
	\item 3 issues referentes a erros reais;
	\item 1 issues provenientes de testes negativos intencionais.
\end{itemize}

\par Isso evidenciou a capacidade da suíte em fornecer documentação automatizada para cada falha detectada.

\subsection{Benefícios diretos para o time}
\par Entre os benefícios concretos, destacam-se:
\begin{itemize}
	\item maior rastreabilidade entre falha, evidência, diagnóstico e correção;
	\item histórico organizado para auditoria e melhoria contínua;
	\item eliminação do trabalho manual de documentação;
	\item agilidade no fluxo de manutenção.
\end{itemize}




\section{Discussão sobre o Uso do Robot Framework para Automação Funcional}
\label{sec:robotframework}

\par O Robot Framework demonstrou ser uma ferramenta adequada para automação funcional devido à sua sintaxe simples, modularidade e grande ecossistema de bibliotecas.

\subsection{Pontos fortes}
\par Entre as características positivas, destacam-se:
\begin{itemize}
	\item sintaxe altamente legível;
	\item separação clara entre lógica, keywords e testes;
	\item suporte nativo a logs e relatórios automáticos;
	\item integração fácil com Selenium, Requests e APIs externas;
	\item compatibilidade com CI/CD.
\end{itemize}

\subsection{Pontos fracos}
\par Algumas limitações também foram identificadas:
\begin{itemize}
	\item forte dependência de espaçamento e formatação;
	\item debugging limitado em comparação com linguagens tradicionais;
	\item dependência da estabilidade do Selenium WebDriver;
	\item necessidade de maior controle em execuções headless.
\end{itemize}

\subsection{Benefícios reais para times e escala}
\par O uso do Robot Framework possibilitou:
\begin{itemize}
	\item aumento da produtividade no desenvolvimento de testes;
	\item redução da curva de aprendizado para novos integrantes do time;
	\item manutenção facilitada pela modularidade das keywords;
	\item escalabilidade do conjunto de testes;
	\item integração fluida com processos de CI/CD e ferramentas de rastreabilidade.
\end{itemize}

