\chapter{Projeto de Teste Automatizado}
\label{cap:testAut}

\par O projeto em questão, visa demonstrar a aplicação pratica do \textit{Robot Framework}\footnote{\url{https://robotframework.org/}} na automação de testes funcionais em uma aplicação \textit{web} real. A base do trabalho  é avaliar como a utilização de um \textit{framework} de automação baseado em \textit{keywords} pode contribuir para a melhoria do processo de garantia da qualidade de software, ao proporcionar maior organização, reutilização de componentes e geração automática de evidências de execução.

%% - - - - - - - - - - - - - - - - - - - - - - - - - - - - - - - - - - -
\section{ Descrição do Projeto}
\label{sec:opcoes}


\par Para a execução dos testes de interface, o projeto faz uso da \textit{SeleniumLibrary}\footnote{\url{SeleniumLibrary}}, que integra o \textit{Robot Framework} ao \textit{Selenium WebDriver}\footnote{\url{https://www.selenium.dev/documentation/webdriver/}}. Essa combinação permite simular o comportamento real de um usuário no navegador. O \textit{Selenium} é um dos \textit{frameworks} mais consolidados para automação de navegadores, oferecendo suporte a diferentes navegadores e sistemas operacionais, o que garante flexibilidade e maior realismo nos cenários de teste.

\par Além do foco na automação de testes, o trabalho busca através da integração com a inteligencia artificial (IA) com a \textit{api} do \textit{gemini}\footnote{\url{https://ai.google.dev/gemini-api/docs}}, gerar um ambiente muito eficiente onde as falhas encontradas durante a execução dos testes não apenas são detectadas, mas também analisadas automaticamente. A IA sugere hipóteses sobre as possíveis causas e até caminhos para uma depuração mais rápida, servindo como um apoio extra para as equipes de \textit{quality assurance} (QA) e de desenvolvimento.

\par Outro ponto importante é a conexão com o \textit{GitHub Issues}\footnote{\url{https://github.com/features/issues}}. Sempre que um defeito é identificado e diagnosticado, ele pode ser registrado automaticamente em um repositório de versionamento. Isso torna o fluxo de trabalho mais ágil e colaborativo, reduzindo atividades manuais e garantindo a rastreabilidade entre o defeito, sua análise e o acompanhamento da correção.

\par Sendo assim, o objetivo deste estudo de caso vai além de apenas validar a execução de casos de teste. Ele busca mostrar o potencial da automação inteligente, unindo o \textit{Robot Framework}, o \textit{Selenium} e recursos de IA para apoiar diagnósticos e tornar o desenvolvimento de software ainda mais orientado à qualidade.

%% - - - - - - - - - - - - - - - - - - - - - - - - - - - - - - - - - - -
\section{Contexto da Aplicação de Testes}
\label{sec:contaplitest}

\par Os testes automatizados desenvolvidos neste projeto foram aplicados sobre uma aplicação web pública e amplamente utilizada em treinamentos de automação: o portal \textit{The Internet Herokuapp}\footnote{https://the-internet.herokuapp.com}. Essa plataforma foi escolhida por oferecer diversos cenários controlados de teste, como formulários de login, \textit{upload} de arquivos, botões dinâmicos e elementos interativos, que permitem validar diferentes aspectos do comportamento de uma aplicação \textit{web}.
\par Durante a execução, o \textit{Robot Framework}, em conjunto com a \textit{SeleniumLibrary}, foi responsável por controlar o navegador e reproduzir as ações do usuário, como preencher campos, clicar em botões e validar mensagens de erro. Esse processo possibilitou avaliar a estabilidade da aplicação e a capacidade do \textit{framework} em detectar e registrar falhas de interface de forma confiável.
\par Além disso, o contexto do teste foi ampliado com a integração de ferramentas complementares. A \textit{API Gemini} foi utilizada para interpretar automaticamente as falhas registradas, fornecendo análises e hipóteses sobre suas possíveis causas, enquanto a integração com o \textit{GitHub Issues} garantiu que cada erro detectado fosse documentado diretamente em um repositório de versionamento, permitindo o acompanhamento e rastreabilidade do ciclo de correção.
\par Dessa forma, o contexto da aplicação de testes reflete um ambiente que simula situações reais de uso e falhas comuns em sistemas web, demonstrando como a automação aliada à inteligência artificial pode contribuir para tornar o processo de validação mais inteligente, ágil e integrado ao fluxo de desenvolvimento de software.

%% - - - - - - - - - - - - - - - - - - - - - - - - - - - - - - - - - - -
\section{Arquitetura e Estrutura do Projeto}
\label{sec:arqestproj}
\par A arquitetura do projeto foi concebida para integrar, de forma contínua e automatizada, diferentes componentes responsáveis pela execução dos testes funcionais, análise inteligente de falhas e abertura de \textit{issues} no repositório de desenvolvimento. Essa integração ocorre dentro de um fluxo de \textit{CI/CD} (Integração Contínua/Entrega Contínua), garantindo que cada nova alteração no código seja validada, analisada e rastreada.

\par A Figura~\ref{fig:bpmn} apresenta uma visão geral do processo, modelado em notação \textit{BPMN} (Business Process Model and Notation). Esse diagrama ilustra o ciclo completo de automação, desde o disparo dos testes até a abertura automática de uma \textit{issue} em caso de falha.

\begin{figure}[H]
	\centering
	\includegraphics[width=0.7\linewidth]{fig/BPMN}
	\caption{Fluxo BPMN da automação proposta}
	\label{fig:bpmn}
\end{figure}

\par O fluxo inicia-se quando o repositório recebe um novo \textit{commit} do usuário, acionando automaticamente a rotina de testes. Além disso, a pipeline também é executada em horários pré-determinados, permitindo verificações periódicas mesmo sem alterações recentes no código. A suíte de testes, implementada em \textit{Robot Framework}, utiliza a biblioteca \textit{SeleniumLibrary} para interação com a interface web, realizando validações funcionais e simulando a rotina de uso do sistema.

\par Em caso de falha, as evidências são capturadas automaticamente (capturas de tela, logs e relatórios detalhados). Esse conjunto de informações é então enviado à \textit{API Gemini}, que analisa o erro, identifica possíveis causas e sugere ações de depuração. Com base nesse diagnóstico, é criada automaticamente uma \textit{issue} no GitHub, contendo título, descrição, evidências e o parecer gerado pela IA.

\par A estrutura modular do projeto, separada em arquivos de teste e arquivos de palavras-chave (\textit{resource files}), permite organização clara e reutilização de comandos. Esse padrão facilita a manutenção, amplia a escalabilidade da suíte de testes e torna o fluxo automatizado mais robusto e confiável.

\par Em síntese, a arquitetura proposta combina ferramentas de automação, inteligência artificial e versionamento de forma integrada, garantindo rastreabilidade completa entre código, testes, falhas, análises e correções.


%% - - - - - - - - - - - - - - - - - - - - - - - - - - - - - - - - - - -
\section{Tecnologias Utilizadas}
\label{sec:tecnol}

\par O desenvolvimento deste projeto envolveu o uso de diversas tecnologias que, em conjunto, permitiram construir um ambiente de automação robusto, integrado e inteligente. Cada uma delas desempenha um papel específico na execução dos testes e na análise automatizada dos resultados.

\subsection{\textit{Robot Framework}}
\label{sub:robot}

\par O \textbf{Robot Framework} é a base do projeto, sendo um \textit{framework} de automação de testes de código aberto amplamente utilizado para testes funcionais, de aceitação e de integração. Seu principal diferencial é o uso de uma linguagem baseada em \textit{keywords}, que torna a escrita dos testes mais legível e acessível, mesmo para profissionais que não possuem conhecimento avançado em programação. Além disso, sua arquitetura modular permite a integração com diferentes bibliotecas e ferramentas externas, o que facilita a expansão das funcionalidades.

\subsection{\textit{Selenium WebDriver} e \textit{SeleniumLibrary}}
\label{sub:selenium}

\par A automação da interface foi implementada por meio da \textbf{\textit{SeleniumLibrary}}, que faz a ponte entre o \textit{Robot Framework} e o \textbf{\textit{Selenium WebDriver}}. Essa integração permite controlar o navegador da mesma forma que um usuário real, simulando ações como cliques, preenchimento de campos e navegação entre páginas. O \textit{Selenium} é uma das tecnologias mais consolidadas para automação de testes em aplicações \textit{web}, oferecendo suporte a múltiplos navegadores e sistemas operacionais, o que garante flexibilidade e portabilidade aos testes.

\subsection{API Gemini}
\label{sub:gemini}

\par A integração com a \textbf{API Gemini}, da Google, representa o elemento de IA do projeto. Sempre que um defeito é detectado durante a execução dos testes, a IA é acionada para interpretar o contexto do defeito e gerar uma análise automática, sugerindo causas técnicas prováveis e ações de depuração. Essa abordagem agrega um nível de automação cognitiva ao processo, reduzindo o tempo de diagnóstico e apoiando as equipes de QA e desenvolvimento na identificação de problemas.

\subsection{GitHub e GitHub Issues}
\label{sub:github}

\par O \textbf{GitHub} foi utilizado tanto como repositório para controle de versão do código quanto como ferramenta de rastreabilidade de falhas. A integração direta com o \textbf{GitHub Issues} permite que cada defeito identificado seja automaticamente registrado como uma ocorrência documentada, contendo detalhes do teste, a mensagem de erro e a análise gerada pela IA. Esse processo automatizado facilita o acompanhamento das correções e promove maior colaboração entre os times envolvidos.
\par Dessa forma, a combinação dessas tecnologias possibilitou a criação de um ecossistema de automação completo, no qual a execução de testes, a detecção de falhas, a análise inteligente e o registro de ocorrências estão integrados em um único fluxo automatizado e eficiente.


%% - - - - - - - - - - - - - - - - - - - - - - - - - - - - - - - - - - -
\section{Fluxo de Execução dos Testes com IA}
\label{sec:fluxoteste}

\par O fluxo de execução dos testes automatizados foi projetado para demonstrar como a integração entre o \textit{Robot Framework}, a IA (API Gemini) e o \textit{GitHub Issues} pode criar um ciclo de validação e análise contínua, tornando o processo de garantia da qualidade mais inteligente e autônomo.

\subsection{Etapa 1: Inicialização do Teste}
\label{sub:etapa1}

\par A execução se inicia no \textit{Robot Framework}, que orquestra todos os componentes do processo. O ambiente de testes é configurado com as bibliotecas necessárias, como \textit{SeleniumLibrary} (para controle do navegador), \textit{RequestsLibrary} (para comunicação HTTP) e \textit{AllureLibrary} (para geração de relatórios visuais). Em seguida, o \textit{framework} abre o navegador definido e acessa a aplicação alvo — neste caso, o portal \textit{The Internet Herokuapp} — para iniciar os cenários de teste funcionais.

\subsection{Etapa 2: Execução das Ações e Validação}
\label{sub:etapa2}

\par Durante os testes, o \textit{Selenium} simula as interações de um usuário real, realizando ações como preencher formulários, submeter dados e validar respostas da interface. Caso a aplicação se comporte conforme o esperado, o teste é marcado como bem-sucedido e registrado no relatório de execução. No entanto, se uma inconsistência for identificada (como um erro de autenticação ou falha de carregamento), o \textit{framework} aciona automaticamente o módulo de análise com IA.

\subsection{Etapa 3: Análise Automática da Falha com IA}
\label{sub:etapa3}

\par Quando uma falha é detectada durante a execução da suíte de testes, o erro correspondente é capturado e enviado à \textit{API Gemini}, da Google. A execução desses testes ocorre de duas formas: automaticamente a cada novo \textit{commit} realizado pelo usuário no repositório e, adicionalmente, em horários pré-determinados definidos na rotina de integração contínua. Dessa forma, cada falha registrada pode ser associada tanto a uma alteração recente no código quanto a verificações periódicas do estado geral do sistema.

\par A \textit{API Gemini} recebe como entrada o contexto completo do evento, incluindo a mensagem de erro retornada, o nome do teste afetado, o \textit{commit} em execução (quando aplicável) e o autor responsável pela modificação, e gera uma resposta textual contendo:

\begin{itemize}
	\item três possíveis causas técnicas para a falha;
	\item duas sugestões de ações de depuração rápida.
\end{itemize}

\par Essa análise automática permite identificar a origem do problema de forma imediata, reduzindo significativamente o tempo necessário para diagnóstico manual e fornecendo insights técnicos diretamente para a equipe de desenvolvimento. Além disso, a combinação entre disparo por \textit{commit} e execuções agendadas aumenta a cobertura, a rastreabilidade e a confiabilidade do processo de validação contínua.

\subsection{Etapa 4: Registro da Falha no GitHub Issues}
\label{sub:regFalha}

\par Após o retorno da análise da IA, o sistema utiliza a \textbf{API do GitHub} para registrar automaticamente uma \textit{issue} no repositório do projeto. Este registro contém:
\begin{itemize}
	\item título descritivo da falha;
	\item mensagem de erro capturada;
	\item diagnóstico gerado pela IA (causas e ações recomendadas);
	\item informações sobre o commit e o responsável pela execução.
\end{itemize}
\par Essa integração garante rastreabilidade entre o erro ocorrido, sua análise e o acompanhamento da correção, permitindo que o fluxo de trabalho entre QA e desenvolvimento se torne mais colaborativo e eficiente.

\subsection{Etapa 5: Relatórios e Evidências}
\label{sub:relato}

\par Por fim, os resultados completos são consolidados pelo \textbf{Allure Report}, que gera relatórios visuais contendo o status de cada teste, as evidências de execução (como capturas de tela) e o histórico das execuções anteriores. Dessa forma, é possível acompanhar a evolução da qualidade da aplicação ao longo do tempo.

\par Em síntese, o fluxo de execução dos testes combina automação funcional, análise inteligente e rastreabilidade contínua, demonstrando como a união entre \textbf{Robot Framework}, \textbf{Selenium}, \textbf{IA Gemini} e \textbf{GitHub Issues} pode criar um ecossistema de validação moderno, ágil e orientado à qualidade.


